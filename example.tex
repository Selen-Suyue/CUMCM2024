%\documentclass{cumcmthesis}
\documentclass[withoutpreface,bwprint]{cumcmthesis} %去掉封面与编号页,电子版提交的时候使用。
\usepackage[superscript]{cite}
\usepackage{booktabs}
\usepackage{longtable}
\usepackage{float}
\usepackage{graphicx}
\usepackage{float}
\usepackage[framemethod=TikZ]{mdframed}
\usepackage{url}   % 网页链接
\usepackage{subcaption} % 子标题
\title{数学建模论文排版}
 
\begin{document}
\maketitle
\begin{abstract}

	这里写摘要,国赛论文摘要要求是一页最好,不要多也不要太少。

	%\keywords{ 科幻小说 \quad  摘要 \quad 三体  \quad  关键字 \quad 科学边界}
	\noindent{ \textbf{关键词:} Fisher精确检验\quad   多元线性回归\quad 系统聚类 \quad 灰色关联分析\quad}
\end{abstract}


\section{问题重述}


问题重述部分正文
\section{问题分析}
\subsection{问题一的分析}
这里是第一段的内容。


\subsection{问题二的分析}

\subsection{问题三的分析}

\section{基本假设与符号说明}
\subsection{基本假设}
$\bullet$ 假设理论物理跟泵不存在;

$\bullet$ 假设数据中未填写的数据项为 0

$\bullet$ 假设所提供的数据准确无误;

$\bullet$ 不考虑因检验手段等原因对数据值的影响。

\subsection{符号说明}


\section{问题一模型的建立与求解}
针对每批零配件,假定总量为$N$,我们考虑采用异常检测的经典取样方法:序贯概率比检测$SPRT$作为抽检方案。在此之前
我们考虑每次取样的样本量为$D_i$,令单个零件次品与否的布尔值为$x$,考虑其单次试验成功(为次品)概率的期望为$\mu$,则其显然
服从经典的二项分布表示:
\begin{equation}
	\textit{Bern}(x|\mu) = \mu^x (1 - \mu)^{1-x}
\end{equation}
接下来考虑其在样本集上的对数似然函数,针对第$i$次取样$D_i$,对其中的每个样本取到观测$x_1,x_2...x_n$,根据题目要求
样本集中零配件的次品产生事件可认定为相互独立的。则其似然函数可写为:
\begin{equation}
	\mathbf{P}(D_i|\mu) = \prod_{n=1}^{N} p(x_n|\mu) = \prod_{n=1}^{N} \mu^{x_n} (1 - \mu)^{1-x_n}
\end{equation}
为便于后续处理,我们取其对数似然:
\begin{equation}
	\begin{split}
		&\ln\mathbf{P}(D|\mu) = \ln \prod_{n=1}^{N} \mu^{x_n} (1 - \mu)^{1-x_n}
		= \ln \mu \sum_{n=1}^{N} x_n + \ln(1 - \mu) \sum_{n=1}^{N} 1 - x_n \\
		&=  \ln \mu \sum_{n=1}^{N} x_n + \ln(1 - \mu) (N - \sum_{n=1}^{N} x_n)
		= \sum_{n=1}^{N} x_n \ln \mu + (1 - x_n) \ln(1 - \mu)
	\end{split}
\end{equation}
接下来我们依据题干给定零假设和备择假设:
\begin{equation}
	\begin{cases}
		H_0: \mu > 0.1 \\
		H_1: \mu \le 0.1
	\end{cases}
\end{equation}
题干中的两种情况意味着拒真和纳伪的显著性水平$\alpha$和$\beta$分别为0.05和0.1。在\textit{SPRT}语境下,考虑决策边界:
$$ A = \ln \frac{\beta}{1 - \alpha} \ \ \ \  B = \ln \frac{1 - \beta}{\alpha}$$
于是,针对每次采样$D_i$,我们需要求出在零假设和备择假设下的似然比$LR$:
\begin{equation}
	LR= \frac{\sum_{n=1}^{N} x_n \ln \mu_0 + (1 - x_n) \ln(1 - \mu_0)}{\sum_{n=1}^{N} x_n \ln \mu_1 + (1 - x_n) \ln(1 - \mu_1)}
\end{equation}
需要注意的是,在原生的$SPRT$场景中,$H_0$和$H_1$一般被认定为较为复杂的参数估计$\theta_0$和$\theta_1$,这取决于它们事先假定样本服从一个较为严谨且高度可表达的
概率分布。然而基于问题一,在没有明确历史数据和概率分布的先验情况下,我们只能将其建模为一般二项分布,
为了遵循$SPRT$的使用场景,我们将二项分布参数建模为$\mu_0=0.1+\Delta \mu$ , $\mu_1=0.1-\Delta \mu$。通过轻微扰动量来拟合样本的分布与所报标称值
的差异,扰动量的设置取决于样本量的大小,这点我们将在后续给出实验和说明。

尽管在许多场景中单样本取样策略以及被证明取得了很好的效果,但考虑到题干背景,我们依然选择样本集作为采样标准。遵循$SPRT$方法,给定总零配件量$N$,初次取样
$D_i$应为按照标称值所取的总样本配比,即$D_1=0.1N$,而后计算出当前样本下的对数似然比$LR_1$。序贯检验比方法遵循以下停止法则:
\begin{equation}
	\gamma = \inf \left\{ n | n \geq 1, LR_n \in (A, B) \right\}
\end{equation}
具体来说,若$LR_1 \le A$,接受$H_0$假设;若$LR_1 \ge B$,接受$H_1$假设,否则继续采样。初次采样的样本量为$D_1=0.1N$,假定每次
采样的次品数为$n_i$,则此后每次采样量依据以下法则确定:
\begin{equation}
	D_{i+1}=D_i-n_i
\end{equation}
检验的完整流程可以作出如下表示:
\end{document}
