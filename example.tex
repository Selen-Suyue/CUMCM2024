%\documentclass{cumcmthesis}
\documentclass[withoutpreface,bwprint]{cumcmthesis} %去掉封面与编号页,电子版提交的时候使用。
\usepackage[superscript]{cite}
\usepackage{booktabs}
\usepackage{longtable}
\usepackage{float}
\usepackage{graphicx}
\usepackage{float}
\usepackage[framemethod=TikZ]{mdframed}
\usepackage{url}   % 网页链接
\usepackage{subcaption} % 子标题
\title{数学建模论文排版}
 
\begin{document}
\maketitle
\begin{abstract}

	这里写摘要,国赛论文摘要要求是一页最好,不要多也不要太少。

	%\keywords{ 科幻小说 \quad  摘要 \quad 三体  \quad  关键字 \quad 科学边界}
	\noindent{ \textbf{关键词:} Fisher精确检验\quad   多元线性回归\quad 系统聚类 \quad 灰色关联分析\quad}
\end{abstract}


\section{问题重述}
某电子产品的生产企业需要综合诸多考虑购置零部件、产品抽检、产品拆解、报废等问题,以确保产品质量的同时降低成本。

\textbf{问题一:}考虑到零配件供应商所述次品率不高于既定标称值,企业拟采用抽样检测方法以验收此批零配件。因为企业寻承担检测费用,企业希望应用数学模型得到最少抽检次数的抽样方案。

已知标称值为 10\%,结合以下两种不同情况,分别设计出具体的抽样检测方案:

1. 拒收条件:在95\%的置信水平下,如果检测结果表明零配件的次品率超出了标称值,那么这批零配件将被拒收。

2. 接收条件:在90\%的置信水平下,如果检测结果表明零配件的次品率未超过标称值,那么这批零配件将被接收。

\textbf{问题二:}在已知零配件及成品次品率情况下,在电子产品生产的零配件检测、装配、成品检测、不合格品拆解的各个阶段为企业作出最优决策。
并且结合判断依据及相应的指标对表1中企业在生产中遇到的情况作出相应的最优决策方案。

\textbf{问题三:}在零配件、半成品和成品的次品率已知情况下,重复问题2的生产决策方案以适配有m道工序、
n个零配件的问题。并且应用此方法针对表2中情况给出判断依据和指标得到最优的决策方案。

\textbf{问题四:}在零配件、半成品和成品的次品率均由抽样检测获得的情况下,重新考虑问题2、3的生产决策方案。
\section{问题分析}
\subsection{问题一的分析}
这里是第一段的内容。


\subsection{问题二的分析}
hahahah
\subsection{问题三的分析}

\section{基本假设与符号说明}
\subsection{基本假设}
$\bullet$ 假设理论物理跟泵不存在;

$\bullet$ 假设数据中未填写的数据项为 0

$\bullet$ 假设所提供的数据准确无误;

$\bullet$ 不考虑因检验手段等原因对数据值的影响。

\subsection{符号说明}


\section{问题一模型的建立与求解}


\end{document}
